\chapter{Funzionamento}

\section{Files}

\subsection{rtt\_check.lua}

Lo script \texttt{rtt\_check.lua} analizza il Round-Trip Time (RTT) dei pacchetti \textbf{TCP e ICMP}. Questi pacchetti vengono esaminati estraendone l'RTT e confrontandolo con i valori di RTT attesi in base alla localizzazione geografica dell'indirizzo IP associato al pacchetto tramite database.\\
\\
Lo script \texttt{rtt\_check.lua} è stato implementato come un \textbf{Post Dissector}\\
\\
Lo script \texttt{rtt\_check.lua} aggiunge una voce personalizzata a ciascun pacchetto contenente:

\begin{itemize}
  \item Il paese reale dell'IP secondo MaxMind: \texttt{Detected country}
  \item Il tempo di risposta misurato (RTT reale): \texttt{Measured RTT}
  \item Il paese stimato in base all'RTT: \texttt{Estimated country}
  \item L'RTT medio stimato per il paese rilevato: \texttt{Expected RTT}
\end{itemize}
\\
Un valore di RTT osservato (\texttt{RTT\_value}) viene considerato \textbf{anomalo} se non rientra nell'intervallo definito dalla media e dalla deviazione standard dei tempi di risposta noti per un determinato paese.\\
\\
Formalmente, un \texttt{RTT\_value} è considerato \textbf{valido} se: 

\begin{equation} 
\texttt{RTT\_MEAN} - k \cdot \texttt{DEVIAZIONE\_STANDARD} \leq \texttt{RTT\_value} \leq \texttt{RTT\_MEAN} + k \cdot \texttt{DEVIAZIONE\_STANDARD} 
\end{equation} 
\\
Nel nostro sistema, il fattore \( k \) è impostato a 2 ( valore comunemente usato per identificare valori anomali in una distribuzione normale ). In realtà, questo valore \( k \) rappresenta un'applicazione dello \textbf{z-score}. In altre parole, \( k \) può essere visto come il valore dello \textbf{z-score} che definisce la soglia di tolleranza per identificare un valore anomalo. Comunemente stabilita come 2 deviazioni standard ( corrispondente a circa il 95\% dei valori in una distribuzione normale ).\\
\\
Pertanto, un RTT è accettabile se si trova \textbf{entro due deviazioni standard dalla media}.\\
\\ 
Se il valore \texttt{RTT\_value} non rientra in questo intervallo, viene etichettato come \textbf{potenziale anomalia} e segnalato come errore di protocollo in rosso su Wireshark.
\\
Questa segnalazione avviene anche in caso di incongruenza tra il \textbf{paese sorgente misurato e il paese sorgente stimato}.

\subsection{Gestione Paesi Sconosciuti e Calcolo del Continente}

All'interno dello script \texttt{rtt\_check.lua}, viene effettuato un confronto tra il paese rilevato per un determinato pacchetto e i dati salvati in memoria nel file \texttt{ntp\_rtt\_stats.txt}, generato dallo script Python.\\
\\
Tuttavia, non tutti i paesi definiti nel database MaxMind sono presenti nel file di riferimento RTT, poiché il file è generato in modo selettivo. Quando ciò accade, lo script opera come segue:

\begin{enumerate}
  \item Estrae le coordinate geografiche ( latitudine e longitudine ) del paese non presente in memoria.
  \item Utilizza queste coordinate per determinare il continente di appartenenza.
  \item Confronta l'RTT del pacchetto osservato con il valore medio stimato per quel continente.
\end{enumerate}
\\
I continenti sono definiti tramite intervalli di latitudine e longitudine che formano dei poligoni complessi. Questi intervalli sono stati verificati attraverso \href{https://www.gps-coordinates.net/}{gps-coordinates.net} e uno script Python di testing, che analizza un set di coordinate geografiche note, rappresentandone i confini:

\begin{figure}[H]
  \centering
  \includegraphics[width=0.9\textwidth]{img/continent_test_output.png}
  \caption{Risultato del test Python per la verifica delle coordinate dei continenti}
\end{figure}

Se si esegue generetor.py dall'italia, il risultato dovrebbe assomigliare ad una cosa di questo tipo:

\begin{figure}[hp!]
  \centering
  \includegraphics[width=0.9\textwidth]{img/Continental_RTT_Avg_Europe.png}
  \caption{RTT medio tra i paesi europei e l'estero}
\end{figure}
\clearpage
\subsection{country.json}

Il sistema sfrutta il file \texttt{country.json}, contenente le associazioni tra codici paese (es. \texttt{"US"}) e i relativi domini dei server \href{https://www.ntppool.org/it/}{NTP} (es. \texttt{0.us.pool.ntp.org}). Questi domini vengono pingati per stimare il tempo medio di risposta.\\

\subsection{parameters.json}

Il file \texttt{parameters.json} contiene i parametri per eseguire i ping:

\begin{itemize}

  \item \texttt {PING COUNT}: definisce quanti ping verranno eseguiti per ciascun host.
  \item \texttt {PING TIMEOUT}: è il tempo massimo di attesa per ogni ping prima di considerarlo fallito.
  \item \texttt {PING INTERVAL}: definisce quanto velocemente i ping vengono inviati dopo aver ottenuto un risultato.
  \item \texttt {OUTLIER FACTOR}: è usato per escludere i valori RTT che sono troppo lontani dalla media, regolando la tolleranza per i valori anomali.
  
\end{itemize}

\subsection{ntp\_rtt\_stats.txt}

Il file di output \texttt{ntp\_rtt\_stats.txt}, generato dal \texttt{generator.py}, contiene le medie degli RTT per ogni paese misurato. Questo viene poi letto dallo script Lua durante l'analisi dei pacchetti.\\

\section{Installazione}

Per iniziare con RTT GAD, segui questi passaggi:

\begin{enumerate}
  \item[\textbf{1}] \textbf{Installa il database GeoIP2 Country di MaxMind} \\
  Scarica il database GeoLite2 Country ( gratuito con registrazione ) o un altro database GeoIP2 Country dal sito ufficiale di MaxMind.\\

  Puoi anche seguire questo utile tutorial video di Chris Greer per maggiori informazioni: \\
  \href{https://www.youtube.com/watch?v=some-video-id}{YouTube – Come scaricare GeoLite2}
  
  \item[\textbf{2}] \textbf{Installa il plugin Lua} \\
  Sposta lo script \texttt{rtt\_check.lua} nella directory dei plugin di Wireshark. Wireshark carica automaticamente i plugin Lua posizionati in queste directory a seconda del tuo sistema operativo:
  
  \begin{itemize}
    \item \textbf{Windows:} C:\textbackslash Program Files\textbackslash Wireshark\textbackslash plugins\textbackslash 
    \item \textbf{macOS:} ~/.local/lib/wireshark/plugins
    \item \textbf{Linux:} /usr/lib/wireshark/plugins/<version>/
  \end{itemize}
  
  Copia lo script Lua (\texttt{rtt\_check.lua}) nella directory dei plugin di Wireshark. Se stai usando la directory dei plugin personali, assicurati che esista e crea la cartella plugins se necessario.
  
  \textbf{Metodo GUI ( consigliato ):} \\
  Apri Wireshark \\
  Clicca su Aiuto → Info su Wireshark \\
  Vai alla scheda Cartelle \\
  Clicca sul link accanto a Plugin Personali \\
  Sposta lo script \texttt{rtt\_check.lua} in questa cartella
  
  \item[\textbf{3}] \textbf{Aggiungi il file di statistiche RTT} \\
  Sposta il file \texttt{ntp\_rtt\_stats.txt} nella directory dei plugin personale di Wireshark come prima. Su windows ls cartella è:

  \begin{itemize}
    \item \textbf{Windows:} C:\textbackslash Users\textbackslash <username>\textbackslash AppData \textbackslash Roaming \textbackslash Wireshark \textbackslash plugins
  \end{itemize}

  \item[\textbf{4}] \textbf{Riavvia Wireshark per caricare il plugin}
  
  \item[\textbf{5}] \textbf{Ora sei pronto per iniziare ad analizzare le anomalie RTT e rilevare incongruenze nella geolocalizzazione!}
\end{enumerate}

\section{Requirements}

Per eseguire lo script \texttt{generator.py}, è necessario avere Python installato sul proprio sistema ( \textbf{Testato su Python versione >= 3.12} ). Lo script utilizza le seguenti librerie standard, che sono incluse nella libreria standard di Python.\\
\begin{itemize}
    \item[\textbf{1}] Non sono richieste dipendenze esterne. 
    \item[\textbf{2}] Questo plugin richiede Wireshark versione 3.6 o successiva.
\end{itemize}
